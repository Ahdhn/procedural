\section{Trabalhos relacionados}
\label{trabalhosRelacionados}

Uma das t�cnicas de modelagem procedural de terrenos � o ru�do Perlin \cite{perlinNoise}, uma fun��o pseudo-aleat�ria que, dado uma entrada (posi��o), retorna um valor que possui uma suave transi��o com os seus vizinhos. Em \cite{improvedPerlinNoise} foi apresentado um ru�do Perlin otimizado, que buscou tornar o ru�do mais amig�vel �s novas arquiteturas (GPUs), melhorar as propriedades visuais e introduzir uma �nica vers�o do ru�do que retornaria os mesmos valores independentemente da plataforma de \emph{hardware} ou \emph{software}.


Em \cite{proceduralApproach} s�o apresentados alguns algoritmos que fazem uso do ru�do Perlin e que s�o capazes de gerar terrenos de uma forma significativamente realista. Podemos citar o algoritmo \emph{fBm}, \emph{heterogenous terrain}, \emph{hybrid multifractal} e \emph{ridged multifractal}, sendo que este �ltimo foi o algoritmo utilizado neste trabalho.

Em \cite{carlucio}, os autores apresentam um paradigma para a modelagem procedural (terrenos, vegeta��o, etc.) utilizando v�rias \emph{threads}. Uma implementa��o � proposta utilizando apenas as unidades de processamento dispon�veis na CPU.

A gera��o procedural utilizando a GPU foi explorada em \cite{generatingComplex} e \cite{Schneider:2006:FractalTerrain}. O primeiro trabalho, faz uso de \emph{geometry shaders} e est� limitado �s placas de v�deo com suporte a DirectX 10. O segundo trabalho, mais abrangente quanto as placas de v�deo suportadas, gera os terrenos na GPU com o uso de algoritmos multifractais (semelhante ao que � proposto aqui). Nenhum dos dois trabalhos, por�m, faz uma compara��o entre implementa��es de gera��o de terrenos utilizando a CPU e a GPU, e tamb�m n�o buscam uma plataforma que utilize as duas arquiteturas.
